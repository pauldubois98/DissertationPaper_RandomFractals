% !TeX encoding = UTF-8
% !TeX spellcheck = en_GB
\documentclass[a4paper, 11pt]{article}
%\documentclass[a4paper, 12pt]{article}
\usepackage[utf8]{inputenc}

%margin
\usepackage[margin=2cm]{geometry}
%\usepackage[margin=1in]{geometry}
\usepackage{changepage}
%line spacing
\renewcommand{\baselinestretch}{1.15}
%\renewcommand{\baselinestretch}{1.6}
%lists
\usepackage{enumitem}
%multi-columns
\usepackage{multicol}
%encoding
\usepackage[utf8]{inputenc}
\usepackage[T1]{fontenc}
%English
\usepackage[english]{babel}
%colors
\usepackage[dvipsnames,svgnames,table]{xcolor}
%tables
\usepackage{multirow}
\usepackage{longtable}
%for math
\usepackage{amsfonts}
\usepackage{amssymb}
\usepackage{mathrsfs}
\usepackage{amsmath}
\usepackage{amsthm}
\usepackage{mathtools}
\usepackage{nicefrac}
\usepackage{makecell}
%for images
\usepackage{graphicx}
\graphicspath{{./img/}{./illustrations/}}
%for plots
\usepackage{tikz}
\usepackage{adjustbox}
%links
%\usepackage[hyphens]{url}
%\usepackage{hyperref}
%\hypersetup{colorlinks=true}
%citations
\bibliographystyle{unsrtnat}
%code
\usepackage{minted}
%\usepackage[cache=false]{minted}
%\usepackage{listings}
\usepackage{algpseudocode}
\definecolor{CodeColor}{RGB}{100, 100, 100}


%environments
\newtheorem*{notation}{Notation}
\newtheorem{definition}{Definition}[section]
\newtheorem*{claim}{Claim}
\newtheorem{proposition}{Proposition}[section]
\newtheorem{property}{Property}[section]
\newtheorem{lemma}{Lemma}[section]
\newtheorem{theorem}{Theorem}
\newtheorem{corollary}{Corollary}[section]
\newtheorem*{conjecture}{Conjecture}
%commands
%\newcommand{\name}[num]{definition}
\newcommand{\primes}{\mathbb{P}}
\newcommand{\N}{\mathbb{N}}
\newcommand{\Z}{\mathbb{Z}}
\newcommand{\Q}{\mathbb{Q}}
\newcommand{\D}{\mathbb{D}}
\newcommand{\R}{\mathbb{R}}
\newcommand{\C}{\mathbb{C}}
\newcommand{\F}{\mathbb{F}}
\newcommand{\halfplane}{\mathbb{H}}
%\newcommand{\dim}[1]{\text{dim}(#1)}
\newcommand{\SL}[2]{\text{SL}_{#1}(#2)}
\newcommand{\Norm}[2][]{\text{Norm}_{#1}(#2)}
\newcommand{\floor}[1]{\lfloor #1 \rfloor}
\newcommand{\ceil}[1]{\lceil #1 \rceil}
\newcommand{\abs}[1]{| #1 |}
\newcommand{\curt}[1]{\sqrt[3]{#1}}
\newcommand{\Ker}[1]{\text{Ker}(#1)}
\newcommand{\Image}[1]{\text{Im}(#1)}
\newcommand{\Gal}[1]{\text{Gal}(#1)}
\newcommand{\Frob}[2]{\text{Frob}_{#1}(#2)}
\newcommand{\Tr}[1]{\text{Tr}(#1)}
\newcommand{\Det}[1]{\text{Det}(#1)}
\newcommand{\End}[1]{\text{End}(#1)}
\newcommand{\Aut}[1]{\text{Aut}(#1)}
\newcommand{\legendre}[2]{\left( \frac{#1}{#2} \right)}
\newcommand{\degree}[1]{\partial #1}
\newcommand{\diam}[1]{\text{diam}(#1)}




\begin{document}
	\include{title_page}
	\begin{center}
		\LARGE Random Fractals
	\end{center}
	\vspace{0.5cm}
	\begin{abstract}
		[blah]
	\end{abstract}
	

	\vspace{2.5cm}
	\begin{center}
		\textbf{Acknowledgment}
	\end{center}
	\vspace{1cm}
	
	First of all, I would like to thank my supervisor, Ben Hambly, for guiding and supporting through this project.
		
	I would like to thank the whole administration team of Oxford, for all the help they provided despite the pandemic context.	
	
	
	\newpage
	
	\tableofcontents
	
	% !TeX spellcheck = en_GB
\setcounter{section}{-1}
\section{Introduction}


	% !TeX spellcheck = en_GB
\section{Background Theory}


%%%%%%%%%%%%%%%%%%%%%%%%%%%%%%%%%%%%%%%%%%%%%%%%%%%%%%%%%%%%%%%%%%%%%%%%%%%%%%%%
\subsection{Dimensions}
The concept of dimension is quite intuitive from a day-life perceptive.
However, the mathematical concept is more involved.
From the non-mathematical world, this can be used to have better understanding of biology, as DNA segments are crooked enough be considered as object with dimension greater than one.

\subsubsection{Intuition}
Some object that we are used to work with have a very commonly admitted dimension:
\begin{itemize}
	\item \textbf{Empty set} / \textbf{Point}: dimension 0
	\item \textbf{Curve} (e.g.: \textit{line}): dimension 1
	\item \textbf{Surface} (e.g.: \textit{square}): dimension 2
	\item \textbf{Volume} (e.g.: \textit{cube}): dimension 3
	\item \textbf{$N$-dim space} (or \textit{$N$-dimensional cuboid}): dimension $N$
\end{itemize}
All of these usual objects have integral dimensions, making it (relatively) easy to understand.

The rule of thumb to calculate the dimension is to double (or, in general, multiply by $n$) the size of the object, and count the number of copies of the original object obtained.
If there is $N$ original objects, the dimension is $\frac{ln(N)}{ln(n)}$.

Some objects have a more complicated dimension (in fact, a non-integral one):
\begin{itemize}
	\item \textbf{Cantor Set}: dimension $log_3(2) = \frac{ln(2)}{ln(3)}$
	\item \textbf{Koch Snowflake}: dimension $log_3(4) = \frac{ln(4)}{ln(3)}$
	\item \textbf{Sierpiński Triangle}: dimension $log_2(3) = \frac{ln(3)}{ln(2)}$
\end{itemize}
The dimension is much less intuitive for these objects, and it justifies creating a formal mathematical definition.

After this quick overview, 3 properties seem desirable for a definition of dimension \cite{Pollicott_LFDT}.
For a set $X$ ($\subset \R^n$, in general):
\begin{enumerate}
	\item If $X$ is a manifold, dimension coincide with the natural preconception.
	\item In some cases, $X$ may have a fractional (i.e. non-integral) dimension.
	\item If $X$ is countable, then $X$ has dimension $0$.
\end{enumerate}
There are several definition for dimension, satisfying different properties.

\subsubsection{Topological Dimension}
\paragraph{Definition}

\paragraph{Properties}

\subsubsection{Box Dimension}
\paragraph{Definition}

\paragraph{Properties}

\subsubsection{Hausdorff Dimension}
\paragraph{Definition}

\paragraph{Properties}

\subsubsection{Relation Between Dimensions}

%%%%%%%%%%%%%%%%%%%%%%%%%%%%%%%%%%%%%%%%%%%%%%%%%%%%%%%%%%%%%%%%%%%%%%%%%%%%%%%%
\subsection{Fractals}

\subsubsection{Definitions}
Fractals are mathematical objects that have been studied since the 17th century.
Generally having a recursive self-similarity pattern, a fractal is defined as a subset of an Euclidean space with non-integral Hausdorff dimension.
 exceeding its topological dimension.


\subsubsection{Famous Examples}


	% !TeX spellcheck = en_GB
\section{Fractal Percolation}
This is the main object we intend to study.

\subsection{Definition}
We will begin with the 2D case, to as it is the most intuitive.
	% !TeX spellcheck = en_GB
\section{Numerics}

	
	\bibliography{references}
	
	\appendix
	\clearpage\null\newpage
\begin{titlepage}
	\newcommand{\HRule}{\rule{\linewidth}{0.5mm}}
		\begin{center}
		\includegraphics[scale=0.08]{oxford_logo.png}
		\vspace*{1cm}
		
		\textsc{\LARGE University of Oxford}\\[0.75cm]
		\textsc{\LARGE Mathematical Institute}
		
		\vspace{1.5cm}
		
		\HRule
		\vspace{0.75cm}
		
		\textit{\LARGE Random Fractals}
		\vspace{0.5cm}
		
		\textbf{\huge Appendix}
		
		\vspace{0.5cm}
		\HRule
		
		\vspace{1.5cm}
		
		\begin{minipage}{0.4\textwidth}
			\begin{flushleft}
				\large
				\textit{Author}\\
				Paul \textsc{Dubois}
			\end{flushleft}
		\end{minipage}
		~
		\begin{minipage}{0.4\textwidth}
			\begin{flushright}
				\large
				\textit{Supervisor}\\
				Ben \textsc{Hambly}
			\end{flushright}
		\end{minipage}
		
		\vfill
		
		{\large April 26, 2021}
	\end{center}
\end{titlepage}
	% !TeX spellcheck = en_GB
\section{Plots}

	% !TeX spellcheck = en_GB
\section{Codes}
%algo
%implementation choices
%links to git hub

	
\end{document}
