% !TeX spellcheck = en_GB
\section{Background Theory}


%%%%%%%%%%%%%%%%%%%%%%%%%%%%%%%%%%%%%%%%%%%%%%%%%%%%%%%%%%%%%%%%%%%%%%%%%%%%%%%%
\subsection{Dimensions}
The concept of dimension is quite intuitive from a day-life perceptive.
However, the mathematical concept is more involved.
From the non-mathematical world, this can be used to have better understanding of biology, as DNA segments are crooked enough be considered as object with dimension greater than one.

\subsubsection{Intuition}
Some object that we are used to work with have a very commonly admitted dimension:
\begin{itemize}
	\item \textbf{Empty set} / \textbf{Point}: dimension 0
	\item \textbf{Curve} (e.g.: \textit{line}): dimension 1
	\item \textbf{Surface} (e.g.: \textit{square}): dimension 2
	\item \textbf{Volume} (e.g.: \textit{cube}): dimension 3
	\item \textbf{$N$-dim space} (or \textit{$N$-dimensional cuboid}): dimension $N$
\end{itemize}
All of these usual objects have integral dimensions, making it (relatively) easy to understand.

The rule of thumb to calculate the dimension is to double (or, in general, multiply by $n$) the size of the object, and count the number of copies of the original object obtained.
If there is $N$ original objects, the dimension is $\frac{ln(N)}{ln(n)}$.

Some objects have a more complicated dimension (in fact, a non-integral one):
\begin{itemize}
	\item \textbf{Cantor Set}: dimension $log_3(2) = \frac{ln(2)}{ln(3)}$
	\item \textbf{Koch Snowflake}: dimension $log_3(4) = \frac{ln(4)}{ln(3)}$
	\item \textbf{Sierpiński Triangle}: dimension $log_2(3) = \frac{ln(3)}{ln(2)}$
\end{itemize}
The dimension is much less intuitive for these objects, and it justifies creating a formal mathematical definition.

After this quick overview, 3 properties seem desirable for a definition of dimension \cite{Pollicott_LFDT}.
For a set $X$ ($\subset \R^n$, in general):
\begin{enumerate}
	\item If $X$ is a manifold, dimension coincide with the natural preconception.
	\item In some cases, $X$ may have a fractional (i.e. non-integral) dimension.
	\item If $X$ is countable, then $X$ has dimension $0$.
\end{enumerate}
There are several definition for dimension, satisfying different properties.

\subsubsection{Topological Dimension}
\paragraph{Definition}

\paragraph{Properties}

\subsubsection{Box Dimension}
\paragraph{Definition}

\paragraph{Properties}

\subsubsection{Hausdorff Dimension}
\paragraph{Definition}

\paragraph{Properties}

\subsubsection{Relation Between Dimensions}

%%%%%%%%%%%%%%%%%%%%%%%%%%%%%%%%%%%%%%%%%%%%%%%%%%%%%%%%%%%%%%%%%%%%%%%%%%%%%%%%
\subsection{Fractals}

\subsubsection{Definitions}
Fractals are mathematical objects that have been studied since the 17th century.
Generally having a recursive self-similarity pattern, a fractal is defined as a subset of an Euclidean space with non-integral Hausdorff dimension.
 exceeding its topological dimension.


\subsubsection{Famous Examples}

